\documentclass{article}
\usepackage[utf8x]{inputenc}
\usepackage[T1]{fontenc}
\usepackage[english]{babel}
\usepackage{lmodern}
\usepackage{amsmath}
\usepackage[pdftex]{graphicx} 
\usepackage{geometry}
\usepackage{charter}
\usepackage{todonotes}



\title{INFO-F-409 -- Learning dynamics\\Article group 7 Review}
\date{13/01/2016}
\author{Jérome Bastogne, Maxime Desclefs, Simon Picard\\\textsc{ULB}}

\begin{document}
\maketitle

\section{Does the introduction explain clearly the content of the paper}
The introduction begins with a quick analogy of what shape could cooperative agreement can take in real life. It specifies the problems that arise with this methods in real life, it is good because those problems will also be present in the simulation. Then introduction then add the apology-forgiveness mechanism to the analogy.\\
Through this example the introduction cleverly state how the paper is structured. Indeed, it first talks about the basic principle of agreement cooperation, then the possible exploit it induces and finally how to resolve them with apology.\\
Finally, the introduction gives an example of where this kind of work has already been studied and conclude with a the straightforward aim of the paper.\\

It might be welcome to introduce the evolutionary aspect of the work here


\section{whether there is sufficient background information to understand the relevance of the work}
The problem of cooperation loss is clearly stated and proved therefore it is easy to understand the relevance of the work.


\section{whether the methods are clearly explained (can the results be reproduced?)}
The model is overall greatly explained.\\
One might want to review those points :
\begin{itemize}
\item In the simulation, the paper say that the population is initialized with a random strategies but is not said which distribution it uses.
\item There is Lack of coherence in the notation of the variable $\mathbf{q}_{c, ij} = \mathbf{q}_{c}^{ij}$
\item Some notation are not explained $p^{k}_{\alpha, ij}$ is the $k^{th}$ element of $\mathbf{p}_{\alpha,ij}$
\item $N$ is not defined
\item How a stationary distribution of strategies works is missing.
\item In general, variable are defined but what they represent is not said.
\end{itemize}
Besides those small issues, the paper precisely define how to model was created and how the results were obtained.


\section{whether the results answer the questions asked in the paper.}
Results cover all aspects of the evolution of individuals behaviour. It analyses all cases where it favorises then defectors, then revenge,... \todo{pas compris la phrase}
The results show clearly what is the general behaviour in each case.\\


The point is then to see how efficient are apologies and how they promote the cooperation. In the results section, one can see that apologies improve the cooperation but moreover, the paper identifies possible problems coming from it. \\
Two deviation are possible if the apology cost is bad. If it is too low or too high it leads to a trust loss.\\
In order to avoid this exploit, the paper gives a policy to follow : $c<\gamma<\sigma$.\\

The result discussion could be improved by explaining more extensively why the graphs is figure 1 use this, at first sight, strange $y$ value.

\section{whether all questions are answered}
There is one and  it is.


\section{whether the conclusion is sufficient}

They discuss clearly the results answering their first questions. 
The paper also makes the correspondence with real life cooperation and compare the results obtained with the meaning of the reactions of real humans.

Missing : No real conclusion about revenge. Should clearly say if revenge was good or bad and give a conclusion on it as it did for forgiveness.


\section{and whether the overall style is ok and}
The paper is accurately written, using academic vocabulary and coherent use of tenses.\\
Since it is always possible to do better, one might take those points into consideration :
\begin{itemize}
\item It is not said how to read the values in tables.
\item The graphs in figure 1 and 2 are rather small.
\item The order of the graphs in figure 1 and 2 is modified which is confusing
\item On figure 3, the y axis is labelled revenge where it should be proportion of revenge. 
\item "the cooperation level is higher than for when no commitment is allowed" is awkward.
\item Table 1 / Table 2 : one should be theoretical results but it is not clearly stipulated, better titles needed. \todo{trop agressif}
\end{itemize}


\section{whether you believe things are missing in the discussion.}

Future work is clearly stipulated and seem interesting. Adding trust lists and sharing them is something humans could do.
The study of evolutionary trust-levels attributed to each individual can be an interesting work.


\section{etc.}


\section{3 positive points concerning the work, clearly specifying why you think they are well-done or interesting}

Everything is clearly stipulated and the work could be done just by reading this paper. 
The work is well divided and we don't go west by reading the paper. The conducting wire is smartly well done.
The results are shown in a way that they directly answer to the preliminary questions/thoughts.
The paper gives us a bit of redundancy that helps the reader to understand it.


\section{3 negative points, which may include missing/unclear explanations or suggestions for improvement}

I think the work is not based on enough references. It makes me think the paper may contain some errors or some gaps.
The conclusion is missing some points like a review of revenge.
Table 1 / Table 2 : one should be theoretical results but it is not clearly stipulated, better titles needed.


\section{at least 3 clear and relevant questions on the content or the methods used which can be asked (next to other questions)}

3)at least 3 clear and relevant questions on the content or the methods used which can be asked (next to other questions). 

Can you give a good and precise example of a practical use of this experiment in real life ? 

There is no experiment in which cooperation converges ? Why ? Was it intended ? 



\end{document}